\documentclass[a4paper, 12pt]{article}
\usepackage[scale = 0.75]{geometry}
\usepackage{bbold}
\usepackage{amsmath}
\usepackage{mathtools}
\usepackage{amsthm}
\theoremstyle{definition}

\usepackage{amssymb}
\usepackage{enumitem}
\newtheorem{example}{Example}
\newtheorem{theorem}{Theorem}
\newtheorem{intuition}{Intuition}
\title{Note on Utility Functions of Lexicographic Preferences}
\date{}
\author{Wenjun Zheng}

\newcommand{\summation}[2]{\sum_{#1}^{#2}}
\newcommand{\minus}{\textbackslash}
\begin{document}
\maketitle
\begin{abstract}
In this note, we provide the condition for the existence of utility functions for lexicographic preferences. To be specific, for $X_i\subset\mathbb{R}$, the lexicographic preferences admit utility representation on $X_1\times X_2\dots\times X_n$ if and only if $X_1,\dots,X_{n-1}$ are countable. 
\end{abstract}

\textbf{Keywords: }lexicographic preferences; utility representation

\section{Introduction}
Lexicographic preferences derive from reality. A typical example is Chinese College Application. Student should first choose the order of universities and then decide their desiring majors. We may know that lexicographic preferences don't admit utility representation, however this argument de facto suffers fallacies. Consider the following two examples.
\begin{example}[Lexicographic preference on $\mathbb{N}\times\mathbb{R}$]
Consider lexicographic preference on $\mathbb{N}\times\mathbb{R}$, then it has utility function
\[
u(x_1,y_1)=x_1+\frac{\arctan(y_1)}{\pi}.
\]
\begin{proof}
Consider pairs $(x_1,y_1)$ and $(x_2,y_2)$ with $x_2>x_1$. Then,
\begin{align*}
u(x_2,y_2)-u(x_1,y_1)&=x_2+\arctan(y_2)-x_1-\arctan(y_1)\\
&\geq x_2-\frac{\frac{\pi}{2}}{\pi}-x_1-\frac{\frac{\pi}{2}}{\pi}\\
&=x_2-x_1-1\geq 0.
\end{align*}
\end{proof}
\end{example}

\begin{example}[Lexicographic preference on more fancy setting]
Consider lexicographic preference on $([0,1]\cap\mathbb{Q})\times [0,1]$, then for all $(x,y)\in([0,1]\cap\mathbb{Q})\times [0,1]$ it has utility function
\[
u(x,y) = \summation{k=1}{l-1}2^{-k}+\frac{2^{-l}}{2}y,\text{ where }x^l=x.
\]
where $[0,1]\times\mathbb{Q}=\{x^1,x^2,\dots\}$.
\begin{proof}
We discuss it into three cases.
\begin{enumerate}[label = (\roman*)]
\item if $(x_1,y_1)=(x_2,y_2)$, then $u(x_1,y_1) = u(x_2,y_2)$;
\item if $x_1>x_2$: let $x^{l_1}=x_1$ and $x^{l_2}=x_2$. 
\begin{align*}
u(x_1,y_1)\geq u(x_1,0)&\geq \summation{k=1}{l_2}2^{-k}\\
u(x_2,y_2)\leq u(x_2,1)&=\summation{k=1}{l_2-1}2^{-k}+2^{-l_2-1}
\end{align*}
clearly, $u(x_1,y_1)\geq u(x_2,y_2)$.
\item if $x_1=x_2$ and $y_1>y_2$, the conclusion is obvious.
\end{enumerate}
\end{proof}
\end{example}

These two examples potentially indicate that for $A\subset\mathbb{R}$ and $B\subset\mathbb{R}$, lexicographic preferences on $A\times B$ might admit a utility representation if $A$ is countable. The next theorem verifies our conjecture.

\section{Result and Intuition}
\begin{theorem}
Consider $X_1,\dots,X_n\subset\mathbb{R}$ and each $X_i$ is infinite. Lexicographic preferences on $X_1\times\dots\times X_n$ admit utility representation if and only if $X_1,\dots,X_{n-1}$ are countable. 
\end{theorem}
\begin{proof}
First, we prove ``only if", then we prove ``if".
\begin{enumerate}[label = (\roman*)]
\item ``$\rightarrow$": proof by contradiction and assume $\succsim$ admits a utility representation. Without loss of generality, suppose $X_i\in\{X_1,\dots,X_{n-1}\}$ is uncountable. Then, consider 
\[
(\bar{x}_1,\dots,\bar{x}_i,a,\dots,\bar{x}_n),\; (\bar{x}_1,\dots,\bar{x}_i,b,\dots,\bar{x}_n), \quad a,b\in X_{i+1}\text{ and }a>b.
\]
Hence,
\[
u(\bar{x}_1,\dots,\bar{x}_i,a,\dots,\bar{x}_n) > u(\bar{x}_1,\dots,\bar{x}_i,b,\dots,\bar{x}_n).
\]
Pick one rational number from interval 
\[
q(\bar{x}_i)\in[u(\bar{x}_1,\dots,\bar{x}_i,b,\dots,\bar{x}_n), u(\bar{x}_1,\dots,\bar{x}_i,a,\dots,\bar{x}_n)]
\]
Further notice that for $\bar{x}_i'>\bar{x}_i$, we have
\begin{align*}
q(\bar{x}_i')&\geq u(\bar{x}_1,\dots,\bar{x}_i',b,\dots,\bar{x}_n)\\
&> u(\bar{x}_1,\dots,\bar{x}_i,a,\dots,\bar{x}_n)\\
&\geq q(\bar{x}_i).
\end{align*}
Consider a mapping from $X_i$ to the set of all $q(x_i)$, where $x_i\in X_i$. It is injection by above argument and surjection by our construction of codomain. Hence, we form a bijection from uncountable set to a countable set\footnote{Infinite subset of a countable set is still countable.}, contradiction yields.
\item ``$\leftarrow$": without loss of generality, we can assume $X_n=\mathbb{R}$. The reason is following: if we can find utility representation on $X_1\times\dots\times\mathbb{R}$, then this utility representation also preserves order on $X_1\times\dots\times X_n$, when $X_n\subset\mathbb{R}$. Let
\[
X=X_1\times\dots\times X_{n-1}\times\mathbb{R}.
\]

Consider set $S=X_1\times\dots\times X_{n-1}\times\mathbb{Q}$. $S$ is countable and  we would like to show $S$ is separable set for $\succsim$. Suppose $x,y\in X\minus S$ and $x\succ y$. There exists $i\in\{1,\dots,n\}$ such that 
\[
x_j=y_j\text{ for all }j<i\text{ and }x_i>y_i.
\]
We can find out $z\in S$ such that $x\succ z\succ y$ by following way:
\begin{enumerate}
\item If there is $z_i\in X_i$ such that $x_i>z_i>y_i$, then we are done. Let $z_j=x_j=y_j$ for all $j<i$ and $z_i=z_i$, and for $k>i$ the value of $z_k$ does not matter;
\item If there is not $z_i\in X_i$ such that $x_i>z_i>y_i$\footnote{This happens when $X_i$ has gap.}, let $z_j=y_j$ for all $j\leq i$. 
\begin{enumerate}
\item Now for $z_{i+1}$, if there is $z_{i+1}\in X_{i+1}$ such that $z_{i+1}>y_{i+1}$, we are done. 
\item If not, let $z_{i+1}=y_{i+1}$.
\item Continue above process until we find some $z_{k}>y_k$ for $k> i+1$.Finally we come to find an element in $z_n\in X_n=\mathbb{Q}$ such that $z_n>y_n$. Such $z_n$'s existence can be guaranteed since we can always find $z_n\in (y_n,y_n+1)\cap\mathbb{Q}$.
\end{enumerate}
\end{enumerate}
In this way, we construct $z$ to separate between $x$ and $y$. Finally, since $\succsim$ on $X_1\times\dots\times\mathbb{R}$ is complete, transitive, and separable, it admits a utility representation. This utility representation preserves order on $X_1\times\dots\times X_n$.
\end{enumerate}
\end{proof}

\begin{intuition}
Think of $\succsim$ on $\mathbb{N}\times\mathbb{R}$. For any pair $(x_1,x_2)$ and $(y_1,y_2)$, we can assigns a greater weight to the first entry and less weight to the second entry. Notice that the least gap between $x_1,y_1$ is $1$, hence let the maximal gap between $x_2,y_2$ less than $1$ will do the trick. Then, an ideal expression will be
\[
u(x_1,x_2)=x_1+\frac{\arctan(x_2)}{\pi}
\]
For $\succsim$ on $\underbrace{\mathbb{N}\times\mathbb{N}\times\dots\times\mathbb{N}}_{n-1}\times\mathbb{R}$, a candidate utility representation will be
\[
u(x_1,x_2,\dots,x_n)=x_1\cdot10^{n-2}+\dots+x_{n-1}\cdot 10^0+\frac{\arctan(x_n)}{\pi}
\]
\end{intuition}

























\end{document}
